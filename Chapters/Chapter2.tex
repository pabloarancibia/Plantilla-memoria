\chapter{Introducción específica} % Main chapter title

\label{Chapter2}

%----------------------------------------------------------------------------------------
%	SECTION 1
%----------------------------------------------------------------------------------------
En este capítulo se describen las herramientas, tecnologías y hardware que se utilizó para el desarrollo del sistema.

\section{Protocolos de comunicación}
\label{sec:protocolos}

\subsection{Protocolo HTTP}

HTTP \citep{WEBSITE:HTTP}, por sus siglas en inglés: \textit{Hypertext Transfer Protocol}, es un protocolo de tipo cliente-servidor \citep{WEBSITE:CLIENTESERVIDOR}, mediante el cual se establece una comunicación enviando peticiones y obteniendo respuestas. 

Las características principales de este protocolo son:
\begin{itemize}
	\item Basado en arquitectura cliente-servidor.
	\item Además de hipertexto (HTML \citep{WEBSITE:HTML}) se puede utilizar para transmitir otro tipo de documentos como imágenes o vídeos.
	\item Es un protocolo de capa de aplicación del modelo OSI \citep{WEBSITE:OSI}.
	\item Se transmite principalmente sobre el protocolo TCP\citep{WEBSITE:TCP}.	
\end{itemize}

HTTP define un conjunto de métodos de petición, cada uno indica una acción a ejecutar en el servidor. Los más utilizados son:
\begin{itemize}
	\item GET: se utiliza para recuperar datos.
	\item POST: sirve principalmente para cargar nuevos datos.
	\item PATCH: este método aplica modificaciones parciales a los datos existentes.
	\item PUT: permite reemplazar completamente un registro.
	\item DELETE: elimina datos específicos.	
\end{itemize}


%Si en el texto se hace alusión a diferentes partes del trabajo referirse a ellas como capítulo, sección o subsección según corresponda. Por ejemplo: ``En el capítulo \ref{Chapter1} se explica tal cosa'', o ``En la sección \ref{sec:ejemplo} se presenta lo que sea'', o ``En la subsección \ref{subsec:ejemplo} se discute otra cosa''.
%
%Cuando se quiere poner una lista tabulada, se hace así:
%
%\begin{itemize}
%	\item Este es el primer elemento de la lista.
%	\item Este es el segundo elemento de la lista.
%\end{itemize}
%
%Notar el uso de las mayúsculas y el punto al final de cada elemento.
%
%Si se desea poner una lista numerada el formato es este:
%
%\begin{enumerate}
%	\item Este es el primer elemento de la lista.
%	\item Este es el segundo elemento de la lista.
%\end{enumerate}
%
%Notar el uso de las mayúsculas y el punto al final de cada elemento.

\subsection{Protocolo MQTT}
\label{subsec:mqtt}

MQTT \citep{WEBSITE:MQTT} son las siglas de \textit{Message Queuing Telemetry Transport}. Se trata de un protocolo de mensajería liviano para usar en casos donde existen recursos limitados de ancho de banda. 

Se transmite sobre protocolo TCP en la arquitectura publish/subscribe \citep{WEBSITE:PUBSUB}.

Los roles que intervienen en un protocolo MQTT son los siguientes:
\begin{itemize}
	\item Publicadores: son los que envían los datos.
	\item Suscriptores: son los que consumen los datos.
	\item Broker: Transmite los mensajes publicados a los suscriptores.	
\end{itemize}

Un cliente puede ser publicador, suscriptor o ambos. El broker es el punto central de la comunicación ya que sin este los mensajes nunca llegarían a destino. 

En la figura \ref{fig:mqttArq} se puede apreciar un ejemplo de comunicación en la arquitectura MQTT.


\begin{figure}[ht]
	\centering
	\includegraphics[scale=.25]{./Figures/mqtt-diagram.png}
	\caption{Arquitectura MQTT publish/subscribe.}
	\label{fig:mqttArq}
\end{figure}

La estructura de mensajes en este protocolo se dividen en dos: \textit{topics} los cuales son de tipo jerárquicos, utilizando la barra (/) como separador, y \textit{payload} en dónde se incluye el mensaje que se quiere transmitir. Por ejemplo: topic: "nodos/procesos/guardar", payload:"mensaje de ejemplo". Siguiendo este ejemplo un cliente podría suscribirse a ese topic o a una jerarquía más alta y recibir todos los mensajes de los topics que comiencen con nodo/procesos. 

\subsection{Eclipse Mosquitto}
\label{subsec:mosquitto}

Eclipse Mosquitto \citep{WEBSITE:MOSQUITTO} es un broker MQTT OpenSource liviano y adecuado para utilizar en todo tipo de dispositivos sobre todo aquellos que cuenten con baja potencia como microcontroladores.
%Se recomienda no utilizar \textbf{texto en negritas} en ningún párrafo, ni tampoco texto \underline{subrayado}. En cambio sí se debe utilizar \textit{texto en itálicas} para palabras en un idioma extranjero, al menos la primera vez que aparecen en el texto. En el caso de palabras que estamos inventando se deben utilizar ``comillas'', así como también para citas textuales. Por ejemplo, un \textit{digital filter} es una especie de ``selector'' que permite separar ciertos componentes armónicos en particular.
%
%La escritura debe ser impersonal. Por ejemplo, no utilizar ``el diseño del firmware lo hice de acuerdo con tal principio'', sino ``el firmware fue diseñado utilizando tal principio''. 
%
%El trabajo es algo que al momento de escribir la memoria se supone que ya está concluido, entonces todo lo que se refiera a hacer el trabajo se narra en tiempo pasado, porque es algo que ya ocurrió. Por ejemplo, "se diseñó el firmware empleando la técnica de test driven development".
%
%En cambio, la memoria es algo que está vivo cada vez que el lector la lee. Por eso transcurre siempre en tiempo presente, como por ejemplo:
%
%``En el presente capítulo se da una visión global sobre las distintas pruebas realizadas y los resultados obtenidos. Se explica el modo en que fueron llevados a cabo los test unitarios y las pruebas del sistema''.
%
%Se recomienda no utilizar una sección de glosario sino colocar la descripción de las abreviaturas como parte del mismo cuerpo del texto. Por ejemplo, RTOS (\textit{Real Time Operating System}, Sistema Operativo de Tiempo Real) o en caso de considerarlo apropiado mediante notas a pie de página.
%
%Si se desea indicar alguna página web utilizar el siguiente formato de referencias bibliográficas, dónde las referencias se detallan en la sección de bibliografía de la memoria, utilizado el formato establecido por IEEE en \citep{IEEE:citation}. Por ejemplo, ``el presente trabajo se basa en la plataforma EDU-CIAA-NXP \citep{CIAA}, la cual...''.
%
%\subsection{Figuras} 
%
%Al insertar figuras en la memoria se deben considerar determinadas pautas. Para empezar, usar siempre tipografía claramente legible. Luego, tener claro que \textbf{es incorrecto} escribir por ejemplo esto: ``El diseño elegido es un cuadrado, como se ve en la siguiente figura:''
%
%\begin{figure}[h]
%\centering
%\includegraphics[scale=.45]{./Figures/cuadradoAzul.png}
%\end{figure}
%
%La forma correcta de utilizar una figura es con referencias cruzadas, por ejemplo: ``Se eligió utilizar un cuadrado azul para el logo, como puede observarse en la figura \ref{fig:cuadradoAzul}''.
%
%\begin{figure}[ht]
%	\centering
%	\includegraphics[scale=.45]{./Figures/cuadradoAzul.png}
%	\caption{Ilustración del cuadrado azul que se eligió para el diseño del logo.\protect\footnotemark.}
%	\label{fig:cuadradoAzul}
%\end{figure}
%
%\footnotetext{Imagen tomada de \url{https://goo.gl/images/i7C70w}}
%
%
%El texto de las figuras debe estar siempre en español, excepto que se decida reproducir una figura original tomada de alguna referencia. En ese caso la referencia de la cual se tomó la figura debe ser indicada en el epígrafe de la figura e incluida como una nota al pie, como se ilustra en la figura \ref{fig:palabraIngles}.
%
%\begin{figure}[htpb]
%	\centering
%	\includegraphics[scale=.3]{./Figures/word.jpeg}
%	\caption{Imagen tomada de la página oficial del procesador\protect\footnotemark.}
%	\label{fig:palabraIngles}
%\end{figure}
%
%\footnotetext{Imagen tomada de \url{https://goo.gl/images/i7C70w}}
%
%La figura y el epígrafe deben conformar una unidad cuyo significado principal pueda ser comprendido por el lector sin necesidad de leer el cuerpo central de la memoria. Para eso es necesario que el epígrafe sea todo lo detallado que corresponda y si en la figura se utilizan abreviaturas entonces aclarar su significado en el epígrafe o en la misma figura.
%
%
%
%\begin{figure}[ht]
%	\centering
%	\includegraphics[scale=.37]{./Figures/questionMark.png}
%	\caption{¿Por qué de pronto aparece esta figura?}
%	\label{fig:questionMark}
%\end{figure}
%
%Nunca colocar una figura en el documento antes de hacer la primera referencia a ella, como se ilustra con la figura \ref{fig:questionMark}, porque sino el lector no comprenderá por qué de pronto aparece la figura en el documento, lo que distraerá su atención.
%
%Otra posibilidad es utilizar el entorno \textit{subfigure} para incluir más de una figura, como se puede ver en la figura \ref{fig:three graphs}. Notar que se pueden referenciar también las figuras internas individualmente de esta manera: \ref{fig:1de3}, \ref{fig:2de3} y \ref{fig:3de3}.
% 
%\begin{figure}[!htpb]
%     \centering
%     \begin{subfigure}[b]{0.3\textwidth}
%         \centering
%         \includegraphics[width=.65\textwidth]{./Figures/questionMark}
%         \caption{Un caption.}
%         \label{fig:1de3}
%     \end{subfigure}
%     \hfill
%     \begin{subfigure}[b]{0.3\textwidth}
%         \centering
%         \includegraphics[width=.65\textwidth]{./Figures/questionMark}
%         \caption{Otro.}
%         \label{fig:2de3}
%     \end{subfigure}
%     \hfill
%     \begin{subfigure}[b]{0.3\textwidth}
%         \centering
%         \includegraphics[width=.65\textwidth]{./Figures/questionMark}
%         \caption{Y otro más.}
%         \label{fig:3de3}
%     \end{subfigure}
%        \caption{Tres gráficos simples}
%        \label{fig:three graphs}
%\end{figure}
%
%El código para generar las imágenes se encuentra disponible para su reutilización en el archivo \file{Chapter2.tex}.
%
%\subsection{Tablas}
%
%Para las tablas utilizar el mismo formato que para las figuras, sólo que el epígrafe se debe colocar arriba de la tabla, como se ilustra en la tabla \ref{tab:peces}. Observar que sólo algunas filas van con líneas visibles y notar el uso de las negritas para los encabezados.  La referencia se logra utilizando el comando \verb|\ref{<label>}| donde label debe estar definida dentro del entorno de la tabla.
%
%\begin{verbatim}
%\begin{table}[h]
%	\centering
%	\caption[caption corto]{caption largo más descriptivo}
%	\begin{tabular}{l c c}    
%		\toprule
%		\textbf{Especie}     & \textbf{Tamaño} & \textbf{Valor}\\
%		\midrule
%		Amphiprion Ocellaris & 10 cm           & \$ 6.000 \\		
%		Hepatus Blue Tang    & 15 cm           & \$ 7.000 \\
%		Zebrasoma Xanthurus  & 12 cm           & \$ 6.800 \\
%		\bottomrule
%		\hline
%	\end{tabular}
%	\label{tab:peces}
%\end{table}
%\end{verbatim}
%
%
%\begin{table}[h]
%	\centering
%	\caption[caption corto]{caption largo más descriptivo}
%	\begin{tabular}{l c c}    
%		\toprule
%		\textbf{Especie} 	 & \textbf{Tamaño} 		& \textbf{Valor}  \\
%		\midrule
%		Amphiprion Ocellaris & 10 cm 				& \$ 6.000 \\		
%		Hepatus Blue Tang	 & 15 cm				& \$ 7.000 \\
%		Zebrasoma Xanthurus	 & 12 cm				& \$ 6.800 \\
%		\bottomrule
%		\hline
%	\end{tabular}
%	\label{tab:peces}
%\end{table}
%
%En cada capítulo se debe reiniciar el número de conteo de las figuras y las tablas, por ejemplo, figura 2.1 o tabla 2.1, pero no se debe reiniciar el conteo en cada sección. Por suerte la plantilla se encarga de esto por nosotros.
%
%\subsection{Ecuaciones}
%\label{sec:Ecuaciones}
%
%Al insertar ecuaciones en la memoria dentro de un entorno \textit{equation}, éstas se numeran en forma automática  y se pueden referir al igual que como se hace con las figuras y tablas, por ejemplo ver la ecuación \ref{eq:metric}.
%
%\begin{equation}
%	\label{eq:metric}
%	ds^2 = c^2 dt^2 \left( \frac{d\sigma^2}{1-k\sigma^2} + \sigma^2\left[ d\theta^2 + \sin^2\theta d\phi^2 \right] \right)
%\end{equation}
%                                                        
%Es importante tener presente que si bien las ecuaciones pueden ser referidas por su número, también es correcto utilizar los dos puntos, como por ejemplo ``la expresión matemática que describe este comportamiento es la siguiente:''
%
%\begin{equation}
%	\label{eq:schrodinger}
%	\frac{\hbar^2}{2m}\nabla^2\Psi + V(\mathbf{r})\Psi = -i\hbar \frac{\partial\Psi}{\partial t}
%\end{equation}
%
%Para generar la ecuación \ref{eq:metric} se utilizó el siguiente código:
%
%\begin{verbatim}
%\begin{equation}
%	\label{eq:metric}
%	ds^2 = c^2 dt^2 \left( \frac{d\sigma^2}{1-k\sigma^2} + 
%	\sigma^2\left[ d\theta^2 + 
%	\sin^2\theta d\phi^2 \right] \right)
%\end{equation}
%\end{verbatim}
%
%Y para la ecuación \ref{eq:schrodinger}:
%
%\begin{verbatim}
%\begin{equation}
%	\label{eq:schrodinger}
%	\frac{\hbar^2}{2m}\nabla^2\Psi + V(\mathbf{r})\Psi = 
%	-i\hbar \frac{\partial\Psi}{\partial t}
%\end{equation}
%
%\end{verbatim}