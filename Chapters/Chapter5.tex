% Chapter Template

\chapter{Conclusiones} % Main chapter title

\label{Chapter5} % Change X to a consecutive number; for referencing this chapter elsewhere, use \ref{ChapterX}


%----------------------------------------------------------------------------------------

%----------------------------------------------------------------------------------------
%	SECTION 1
%----------------------------------------------------------------------------------------

\section{Conclusiones generales }

En este capítulo se detallan cuáles son los principales aportes del trabajo realizado y cómo se podría continuar. 

\subsection{Resultados obtenidos}

Se desarrolló un sistema de seguimiento de procesos para las órdenes de trabajo en una empresa de rectificaciones de motopartes de Resistencia, Chaco. El sistema fue implementado en diciembre de 2022. 

Se logró implementar con éxito un servidor local con todos los servicios desarrollados y de terceros. Además se configuró el acceso remoto para su mantenimiento y actualización.

Se instaló una red local para el sistema con soporte Wi-Fi.

El sistema integral aporta un mejor seguimiento y análisis de los trabajos que se realizan día a día en la empresa.

\subsection{Cumplimiento de los requerimientos}

En el capitulo \ref{sec:objetivosyalcance} se plantearon los requerimientos mínimos que se debían cumplir en el presente trabajo. Al hacer un análisis final del sistema con sus ensayos y resultados obtenidos se concluye que se lograron alcanzar estos objetivos con éxito. 

En cuanto a los riesgos, el único que se presentó fue el no poder cumplir con el tiempo de entrega planificado. En este caso se pudo acordar sin inconvenientes con el cliente la entrega del sistema completo fuera de la fecha prevista. 

\subsection{Modificaciones a lo planificado}
Cuando se presentaron los primeros avances al cliente se realizó una modificación en el sistema a fines de evitar costos de mantenimiento. Se reemplazó el desarrollo de una interfaz pública de consulta de órdenes de trabajo, con acceso desde internet sin autenticación, por el servicio de mensajería detallado en los capítulos \ref{subsec:apimessenger} y \ref{subsec:apimessengerimplementación}. El cliente optó por la opción de mensajería para evitar costos de alojamiento web o de servicios de IP pública y direccionamiento DNS \cite{dns} y para tener un contacto directo con el cliente. 

%\begin{itemize}
%\item ¿Cuál es el grado de cumplimiento de los requerimientos?
%\item ¿Cuán fielmente se puedo seguir la planificación original (cronograma incluido)?
%\item ¿Se manifestó algunos de los riesgos identificados en la planificación? ¿Fue efectivo el plan de mitigación? ¿Se debió aplicar alguna otra acción no contemplada previamente?
%\item Si se debieron hacer modificaciones a lo planificado ¿Cuáles fueron las causas y los efectos?
%\item ¿Qué técnicas resultaron útiles para el desarrollo del proyecto y cuáles no tanto?
%\end{itemize}


%----------------------------------------------------------------------------------------
%	SECTION 2
%----------------------------------------------------------------------------------------
\section{Próximos pasos}

Durante la fase de pruebas y tras la implementación del sistema en producción, se tomaron nota de las posibles mejoras detectadas y se recibió retroalimentación de los usuarios. Con esta información, se elaboró una lista de posibles mejoras, entre las cuales se destacan:

\begin{itemize}
\item Mejora del diseño gráfico para una mejor experiencia e interfaz de usuario.
\item Desarrollo de un nodo móvil con sensor RFID y pantalla integrada.
\item Creación de una sección de administración.
\item Generación de reportes gráficos.
\item Implementación de una sección para la gestión de caja.
\end{itemize}

Asimismo, se tiene previsto la comercialización del sistema en otros entornos y sectores industriales.